\documentclass{article}

\usepackage[french]{babel}
\usepackage[T1]{fontenc}
\usepackage[utf8]{inputenc}

\usepackage{amsmath}
\usepackage{amssymb}
\usepackage{amsthm}
\usepackage{stmaryrd}



\begin{document}

\begin{center}
\large\sc TP 3 - Feuille d'exercices
\end{center}

\section{Exercice 1 :} 

\paragraph{Consigne : } Écrivez une application utilisant les fragments, supportant le format tablette (landscape et screen large), ainsi que les mobiles (tous les autres affichages).

L'application présente une liste de titres d'articles, dans un fragment. Quand on clique sur un titre, un second fragment affiche le contenu de l'article (images, texte).

Vous pouvez stoker les articles "en dure", car le bute de l'exercice est de manipuler les fragments.

En vue "tablette", les deux fragments sont affiché cote à cote.
En vue "téléphone", le fragment "vue de l'article" remplace le fragment "liste des articles".

Nb : Vous pouvez remplacer les articles par des pages de livre, de blog, ou n'importe quel contenu qui vous amuse.

\paragraph{Bonus :} Vous récupérez vos données en ligne depuis un site/blog/wikipedia/etc...


\section{Exercice 2:}
\paragraph{Consigne :} Ajoutez une App Bar à l'exercice précédent. Elle doit disposer d'une icône favoris ($\heartsuit$), qui ajoute l'article visionné dans une liste de favoris. Ajouter un menu "Mes favoris" dans la partie déroulante de la barre, qui ouvre une activité listant les articles favoris, et permettant :
- De supprimer un article marqué comme favoris via un bouton [X].
- D'ouvrir l'article en question (c'est à dire placer l'article dans le fragment chargé d'afficher son contenus) par un simple clic sur son titre.

\end{document}
