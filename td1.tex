\documentclass{article}

\usepackage[french]{babel}
\usepackage[T1]{fontenc}
\usepackage[utf8]{inputenc}

\usepackage{amsmath}
\usepackage{amssymb}
\usepackage{amsthm}
\usepackage{stmaryrd}



\begin{document}
\section{EX 1 :} 

\paragraph{Consigne : } Créer une application avec deux boutons : start et stop, ainsi que deux affichages horaires (temps courant et chronomètre).

Les deux boutons seront sur la même ligne horizontale. Les deux affichages numériques sur une même autre ligne horizontale.
Le premier affichage horaire indique l'heure courante. Le second sera initialisé à 0.

Le bouton start est actif, le bouton stop est inactif (disabled). Un premier appui sur le bouton start rend le bouton stop actif, et désactive le bouton start. Un appui sur le bouton stop active le bouton start et désactive le bouton stop. Il provoque la mise à jour du second compteur horaire qui affiche l'intervalle de temps entre les deux cliques.

\paragraph{Bonus 1 :} Afficher un historique des chronométrages effectués.

\section{EX 2:}
Une application avec 3 boutons, ainsi qu'un imageView. La pression d'un bouton change l'image affichée dans l'imageView. Prenez 3 images différentes, le choix des images est libre.


\paragraph{Bonus 1:} Les boutons ont des vignettes miniature des images à afficher.

\paragraph{Bonus 2:} Un bouton permet de lancer une seconde activité qui ne fait qu'afficher l'image sélectionnée en plein écran.

\paragraph{Bonus 3:} Un bouton supplémentaire permet de prendre une photo et de l'afficher dans l'imageView.

\paragraph{Bonus 4:} Inventez votre propre bonus et vendez le moi.

\end{document}
