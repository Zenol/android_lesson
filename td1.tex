\documentclass{article}

\usepackage[french]{babel}
\usepackage[T1]{fontenc}
\usepackage[utf8]{inputenc}

\usepackage{amsmath}
\usepackage{amssymb}
\usepackage{amsthm}
\usepackage{stmaryrd}



\begin{document}
\section{EX 1 :} 

\paragraph{Consigne : } Créer une application avec deux boutons : start et stop, ainsi que deux affichages horaires.

Les deux boutons seront sur la même ligne horizontal. Les deux affichages numériques sur une même autre ligne horizontal.
Le premier affichage horaire indique l'heure courante. Le second seras initialisé à 0.

Le bouton start est actif, le bouton stop est inactif (disabled). Un premier appuis sur le bouton start rend le bouton stop actif, et désactive le bouton start. Un appuis sur le bouton stop désactive le bouton stop active le bouton start et désactive le bouton stop. Il provoque la mise a jour du second compteur horaire qui affiche l'interval de temps entre les deux cliques.

\paragraph{Bonus 1 :} Le second affichage défile au fur et a mesure que le temps s'écoule, comme un vrais chronomètre.

\section{EX 2:}
Une application avec 3 boutons, ainsi qu'un imageView. Le choix des images est libre. La pression d'un bouton change l'image affiché dans l'image view.


\paragraph{Bonus 1:} Les boutons on des vignettes signature des images à afficher.

\paragraph{Bonus 2:} Un bouton permet de lancer une seconde activité qui ne fait qu'afficher l'image sélectionné en pleine écrans.

\paragraph{Bonus 3:} Un bouton permet de prendre une photo et de l'afficher à la place des 3 autres images.

\paragraph{Bonus 4:} Inventez votre propre bonus et vendez le moi.

\end{document}