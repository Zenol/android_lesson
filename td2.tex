\documentclass{article}

\usepackage[french]{babel}
\usepackage[T1]{fontenc}
\usepackage[utf8]{inputenc}

\usepackage{amsmath}
\usepackage{amssymb}
\usepackage{amsthm}
\usepackage{stmaryrd}



\begin{document}

\begin{center}
\large\sc TP 2 - Feuille d'exercices
\end{center}

\section{Exercice 1 :} 

\paragraph{Consigne : } Écrivez un rapport expliquant comment réaliser une application qui prend une photo, l'enregistre, et la rendre disponible dans la galerie. Vous pouvez vous aider de http://developer.android.com/training/camera/photobasics.html.

Ce doit être un texte lisible, compréhensible, mais vous pouvez illustrer avec quelques cours et brefs extraits de code.
Il faut bien entendu implémenter l'application en question.

Votre rapport doit expliquer les mécanismes que vous utilisez (Intent, appareille photo du téléphone, etc.).


\section{Exercice 2:}
Réaliser un tic tac toe. Le jeux doit comporter un menu d’accueils, un écrans de crédits, et un écrans de jeu.

Elle doit fournir les deux modes de jeux :
\begin{itemize}
\item Un VS Un
\item Un VS IA
\end{itemize}

Le code ainsi qu'un rapport doit être remis. Le rapport doit décrire les fonctionnalités de votre projet, et les choix techniques effectuer.


\paragraph{Bonus 1:} La possibilité de remonter dans le temps et revenir en arrière sur un coup.

\paragraph{Bonus 2:} Un mode de jeu "puissance 4".

\paragraph{Bonus 3:} (Valable uniquement si l'application est parfaite, et les bonus 1 et 2 implémentés) Rendez votre application jolie est attractive.

Ex : Un morpion thématique sur les fleurs, avec des images de tournesol et de tulipe pour les croix/rond. Des boutons entourés de ronces, un menu avec un fond de feuille d'arbre, etc....


\end{document}
