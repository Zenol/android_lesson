\documentclass{beamer}

\usepackage[french]{babel}
\usepackage[T1]{fontenc}
\usepackage[utf8]{inputenc}

\usetheme{Warsaw}
\useoutertheme{infolines}

\usepackage{amsmath}
\usepackage{amssymb}
\usepackage{amsthm}
\usepackage{stmaryrd}

\usepackage[all]{xy}

%Les sous listes on des triangles
\setbeamertemplate{itemize item}[circle]
\setbeamertemplate{itemize subitem}[triangle]
%Les elements caché sont grisé
\beamertemplatetransparentcovered

\begin{document}

\title{Android - Les fondamentaux}
\author{Jérémy S. Cochoy}
\institute{INRIA Paris-Saclay | jeremy.cochoy@u-psud.fr}
\date{Octobre 2015}


\begin{frame}
\titlepage
\end{frame}

\begin{frame}
\tableofcontents
\end{frame}

\begin{frame}
\frametitle{La documentation}

\begin{block}{Votre nouveau livre de chevet.}
\begin{center}
\emph{https://developer.android.com/guide/index.html}
\end{center}
\end{block}

\end{frame}

\section{Intention}

\begin{frame}
\frametitle{Qu'est-ce qu'une intention?}

\begin{block}{Une \verb!Intent!}
Une \verb!Intent! est un objet qui représente un message, et permet de demander une action a un Composant.
\end{block}

\begin{block}{Les usages principaux :}
	\begin{itemize}
		\item Lancer une activité.
		\item Lancer un service.
		\item Délivrer un message.
	\end{itemize}
\end{block}
\end{frame}


\begin{frame}
\frametitle{Démarrer une activité}

\begin{block}{\verb!Activity!}
Une \verb!Activity! représente un écran dans une app. On peux lancer une activité en donnant un \verb!Intent! à \verb!startActivity()!. L'intention décri l'activité et lui fournis les données.
\end{block}

\begin{block}{Avec résultat}
Si l'activité doit fournir un résultat, on dispose de \verb!startActivityForResult()! et de \verb!onActivityResult()!.
\end{block}
\end{frame}

\begin{frame}
\frametitle{Démarrer un service}

\begin{block}{Un service effectue des actions en arrière plan, sans interface.}

\begin{itemize}
\item One time operation : Donner un Intent à \verb!startService()!.
\item Si le service a une interface cleint-serveur, on dispose de \verb!bindService()!.
\end{itemize}
\end{block}
\end{frame}

\begin{frame}
\frametitle{Délivrer un message}

\begin{block}{Broadcast}
Un broadcast est un message que n'importe quelle application peut recevoir.
\end{block}

\begin{block}{Délivrer un message}
Donner un \verb!Intent! à \verb!sendBroadcast()! ou \verb!sendOrderedBroadcast()!.
\end{block}
\end{frame}


\section{Conclusion}

\begin{frame}
\begin{center}
Pour me contacter : jeremy.cochoy@u-psud.fr, merci et à bientôt.

\medskip
\medskip
\medskip
\medskip

\includegraphics[scale=0.18]{android.jpg}
\end{center}
\end{frame}

\end{document}
